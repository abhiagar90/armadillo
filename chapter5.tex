\chapter{Conclusion and Further Work}
%TODO - expand the questions
\section{Conclusion}

To summarize from the results of our queries in the graph, it is clearly seen that the points we claimed previously are indeed true. Clear connections do exist between corporate and political spheres as seen in the networks of Jayant Sinha and Kamal Nath. Moreover the clusters of Birlas and Ambanis suggest that the control and distribution of power is prevalent mostly in close family ties. \\

The work also showed the challenges of collecting and processing data from different sources especially in the Indian context. The absence of proper digitization of the data, adherence to any uniform formats and the absence of good interface for its display all added to the difficulty. \\

As a solution our system provides a good riddance from all above problems to become a common point for connected data access in Indian context. The functionalities thus given are adequate enough to be used by the various users for the purposes they desire. \\

\section{Future work}
%\lipsum[2]
We have tried to start a process of building a system which in the long run will help the Indian society. Our best efforts were to cover up as much functionality as possible. Yet however, many other problems or features are left to be done in future.
%TODO - expand the questions
\begin{itemize}
\item \textbf{Scaling up/out} - The designed system works fine and smoothly for the current amount of data. But question remains how to handle the data storage and processing when more data is pushed? Practically scaling can be done vertically by adding more powerful servers CPU or horizontally by the use of distributed databases. Having said that, a new problem of data remodeling would creep up if the database type is changed.

\item \textbf{Interactive Visualizations} - The visualizations existing presently in the system are very basic with simple interactions. More features can be added to help analysts, journalists to get more insights out of the data. Functionalities should be present to enable people to annotate a visualization, download it in different file formats, interact with it to reveal interesting patterns.

\item \textbf{Better Query Engine} - Side by side with the visualization lies the query engine. As of now it only supports cypher queries as in Neo4j. But provisions can be made to add UI elements in a way such that users can form queries with little or no knowledge of cypher.

\item \textbf{Inference Engine} - Till now it is the humans who are making analysis through the help of the knowledge base and visualizations. But we believe the system can be extended to let the machines draw conclusions from the same. Thinking in terms of Expert Systems and Symbolic AI, the knowledge base can be seen as a set of predicates (entities) and implications (relations). With the help of proper scripts of logic programming, new interesting applications can be made that inferences about the Indian society.

\item \textbf{Better analysis of Social Networks} - The graph can be used to perform in depth social network analysis. Factors such as centrality of the graph can tell about the important entities present. Similarly, weak ties, size of clusters in network can reveal information about lobbying. Also a comparison between our graph with a random graph may lead to discovery of interesting social network patterns.

\item \textbf{Other use cases} - Several other use cases can be incorporated and analyzed with the data. Data about the IAS officers might show the possible relationships between the bureaucracy and the elites. Data about financial contributions of individuals or institutions might show the affiliations of big players with other entities. Moreover, the ownership data of different medias can be used to explain their possible content.

\end{itemize}
